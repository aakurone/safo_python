\documentclass[fleqn,10pt]{olplainarticle}
% Use option lineno for line numbers 
\usepackage{bigints}
\usepackage{amsmath}

\title{Time-averaged intensity of interference from multiple sources}

\author[1]{Antti Kuronen}
\affil[1]{Department of Physics, University of Helsinki, Pietari Kalmin katu 2, Helsinki 00014, Finland}
\keywords{interference, computational physics, Fortran, Python}

\begin{abstract}
This document describes the calculation of time-averaged (over one period) intensity of multiple sources. Used in the course FYS1004 Säteilykentät ja fotonit, University of Helsinki.
\end{abstract}

\begin{document}

\flushbottom
\maketitle
\thispagestyle{empty}

%\section*{Time-averaged intensity of interference from multiple sources}

Amplitude of source $i$ with amplitude $E_i$ and at distance $r_i$ and at time $t$
\begin{equation}
A_i(r_i,t) = E_i \cos(2\pi r_i + 2\pi t)
\end{equation}
Here we have set wavelength and frequency to unity: $\lambda=f=1$.

We need to sum over the sources to get the total amplitude and integrate the square of the total amplitude over one period ($P=1/f=1$) to obtain the time-averaged intensity:

\begin{equation}
I(\mathbf{r}) = \bigintsss_0^1 \left[ \sum_{i=1}^n A_i(r_i,t) \right] ^2 dt = 
\bigintsss_0^1 \left[ \sum_{i=1}^n E_i \cos(2\pi r_i + 2\pi t) \right] ^2 dt 
\end{equation}
Here $r_i$ is now the distance between point $\mathbf{r}$ and the location of source $i$.

Plugging this in \textsc{Maple} the integral can be calculated exactly:
\begin{equation}
I(\mathbf{r}) = 2 (S_1^2 + S_2^2 - S_2 S_3) + \frac{1}{2} S_3^2 
\end{equation}
where
\begin{align}
S_1  & =  \sum_{i=1}^n E_i \sin(\pi r_i)\cos(\pi r_i) \\
S_2  & =  \sum_{i=1}^n E_i\cos^2(\pi r_i) \\
S_3  & =  \sum_{i=1}^n E_i \\
\end{align}

As an example set $n=2$. This gives us 
\begin{equation}
I(\mathbf{r}) = \frac{E_1^2}{2} + \frac{E_2^2}{2} + E_1 E_2 \cos(2\pi r_1 - 2\pi r_2).
\end{equation}
Assuming $E_1=E_2=1$ we get
\begin{equation}
I(\mathbf{r}) = 1 + \cos(2\pi \Delta r),
\end{equation}
where $\Delta r = r_2-r_1$. (This is excatly the expression in \texttt{doublesource\_2dmap.f90}.)

Calculation of the interference pattern is implemented as a Fortran subroutine in source code file \texttt{multisource\_2dmap.f90} where you can set up multiple sources in a 2D square lattice. The file can be compiled as a Python module (for instructions, see the comments in the file) and an usage example is given in \texttt{multisource.py}.

\printbibliography

\end{document}
